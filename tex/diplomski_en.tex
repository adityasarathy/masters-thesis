\documentclass[times, utf8, diplomski, english]{fer}
\usepackage{booktabs}

\begin{document}

% TODO: Navedite broj rada.
\thesisnumber{1417}

% TODO: Navedite naslov rada.
\title{Deep Learning Model for Base Calling of MinION Nanopore Reads}

% TODO: Navedite svoje ime i prezime.
\author{Marko Ratković}

\maketitle

% Ispis stranice s napomenom o umetanju izvornika rada. Uklonite naredbu \izvornik ako želite izbaciti tu stranicu.
\izvornik

% Dodavanje zahvale ili prazne stranice. Ako ne želite dodati zahvalu, naredbu ostavite radi prazne stranice.
\zahvala{Thanks ...}

\tableofcontents
\listoffigures
\listoftables

%%%%%%%%%%%%%%%%%%%%%%%%%%%%%%%%%%%%%%%%%%%%%%%%%%%%%%%%%%%%%%%%%%%%%%%%%%%%%%%%%%%%%%%
%% CHAPTER
\chapter{Introduction}

In recent years,  deep learning and usage of deep neural networks have significantly improved the state-of-the-art in many application domains such as computer vision, speech recognition, and natural language processing. 
In this thesis, we present application of deep learning in the fields of Biology and Bioinformatics for analysis of DNA sequencing data. 

DNA is a molecule that makes up the genetic material of a cell responsible for carrying the information an organism needs to survive, grow and reproduce. 
It is a long polymer of simple units called nucleotides attached together to form two long strands that spiral to create a structure called a double helix. The order of these bases is what determines DNA's instructions, or genetic code.
DNA sequencing is the process of determining this sequence of nucleotides. Originally sequencing was very expensive process but 
during the last couple of decades, the price of sequencing drastically dropped. A significant breakthrough occurred in May 2015 with the release of MinION sequencer by Oxford Nanopore making DNA sequencing inexpensive and available even for small research teams.

Base calling is a process assigning sequence of nucleotides (bases) to the raw data generated by the sequencing device or sequencer. Simply put, it is a process of decoding the output from the sequencer.




\section{Objectives}
Goal of this thesis is to present novel approach for base calling of raw Nanopore sequencing data using deep learning convolutional neural networks.
\section{Organization}
\indent Chapter 2 gives more detailed explanation of the problem, background on nanopore sequencing and overview of state-of-the-art basecallers.

Chapter 3 describe used deep learning concepts in detail used later on in later chapters.

Chapter 4 goes into implementation details, training of the deep learning model and explains methods used to evaluate obtained results. 

Chapter 5 consists of the results of testing performed on different datasets as well as comparison with state-of-the-art basecallers.

In the end, the Chapter 6 gives a brief conclusion and possible future work and improvements of the developed basecaller.

%%%%%%%%%%%%%%%%%%%%%%%%%%%%%%%%%%%%%%%%%%%%%%%%%%%%%%%%%%%%%%%%%%%%%%%%%%%%%%%%%%%%%%%
%% CHAPTER
\chapter{Preliminaries}

\section{Sequencing}
DNA is bla bla. 
sekvenciranje je.... slika, basecalling


SLika ona, bla bla
1. generation sanger
2. generation
3.Generacija blabla


\section{Oxford Nanopore}

\subsection{Technology}

\subsection{Basecalling}




\section{Existing basecallers}

Official basecaller for Nanopore reads is Metrichor. 

MetriKNOW - software that analyses signal as segments it into blocks called events. Events are described with lenght of the block, mean value of messured current and its variance.

Metrichor - uses events output from minknow and models HMM  - each state represents contex in the pore - 5 base pares. When transitioning from one state to an other - event is emited. By modeling emission probabilities and transition probabilities - from sequence of events, using viterbi algorithm is not difficult to determine most probable sequence expressed as series of transitions ih the model.
From states modeled as 5 bp it is impossible to model reppetitions of more than 5 bases.


Nancall - open source basecaller uses HMM approch like the original R7 Metrichor). Metrichor. Supports only R7 chemistry. 
Nanocall [7] was the first open-source basecaller for the MinION offered as an alternative to the proprietary Metrichor software. It was written in C++. Nanocall accepts the segmented signal from minKNOW and assigns
k-mers to the events using a hidden Markov model (HMM). A

DeepNano - first open source basecaller based on neural networks. It uses bidirectional rnn. Originaly supported R7 chemistry, later on support for R9.4 and R9.5 was added.

Deepnano is a python package built on the Theano framework, and uses a deep recurrent neural network (RNN) model to call bases. ONT has also moved towards RNN basecalling and this is now the main method for calling R9 reads

DeepNano Before Metrichor made its own switch from HMM- to RNNbasecalling,
the open-source basecaller DeepNano [8] already implemented a
form of RNN basecalling, booking a significant improvement in accuracy with
respect to the then-current Metrichor version (corresponding to the SQK-MAP-
006 kit, late 2015).
 DeepNano was written in Python, using the Theano library
[50]. The RNN employed in DeepNano consists of 3 hidden layers of 100 units per
layer for 1D basecalling and 4 hidden layers of 250 units for 2D. Rather than
LSTM-nodes, as currently used in Metrichor basecallers, DeepNano implements
gated recurrent units (GRUs) [51] to accou


NanoNet Nanonet provides recurrent neural network basecalling via CURRENNT.

Albacore -Albacore is a C++ project designed to provide a high-performance end-to-end analysis pipeline that can be run on (potentially) any platform. Albacore is currently only available through the ONT Developer Channel to users who have signed the Developer terms and conditions. 

Scrappie - A proprietary basecaller by Metrichor and platform for ongoing
development, Scrappie was the first basecaller reported to specifically address
homopolymer basecalling. It was just recently released on their official GitHub.



%%%%%%%%%%%%%%%%%%%%%%%%%%%%%%%%%%%%%%%%%%%%%%%%%%%%%%%%%%%%%%%%%%%%%%%%%%%%%%%%%%%%%%%
%% CHAPTER
\chapter{Methods}
Thesis Methods.
\section{Arhitecture}
Describe classic arhitecture: convolution, pooling, activation
Zasto svaki dio itd
Batch Normalization 
Iterativno se treniraju.
\subsection{CNN}
\subsection{Residual Networks}
Vanishing gradient problem
\subsection{Gated Residual Networks}
Why not use only needed - distribution of gates

\section{CTC Loss}
\subsection{Definition}
\subsection{Decoding}


%%%%%%%%%%%%%%%%%%%%%%%%%%%%%%%%%%%%%%%%%%%%%%%%%%%%%%%%%%%%%%%%%%%%%%%%%%%%%%%%%%%%%%%
%% CHAPTER
\chapter{Implementation}
\section{Deep Learning model}
\section{Training}
\section{Evaluation methods}
\section{Technologies}

Overall solution was implemented in Python programing language. Described model is implemented using TensorFlow. Tensorflow is an open source software library for numerical computation using data flow graphs developed by Google.  TensorFlow, even tho is considered low-level framework offers implementations of higher level concepts (layers, losses, and optimizers) which makes it great for prototyping while keeping it modular and extensible for highly specific tasks as well.

Tensorflow offers efficient GPU implementations of various layers and losses but as of version 1.2 lacks GPU implementation of used CTC loss, so WARP-CTC (https://github.com/baidu-research/warp-ctc) was used. It offers both GPU and CPU implementations as well as bindings for Tensorflow.

For alignment tasks, developed tool offers support for GraphMap and BWA but can easily be extended with any other aligner that outputs results in Sam file format.

SAMTools(link) and its python bindings PySam(link) were used for conversions between various file formats used in Bioinformatics.

Docker was used for automating the deployments on different machines. It helps us resolve problem know as  "dependency hell"(link)  keeping all dependencies in single container thus eliminating possible conflict between packages on host OS.
Nvidia docker was used for GPU support.

All training was done on the server with  Intel(R) Xeon(R) E5-2640 CPU, 600 GB of RAM and NVIDIA TITAN X Black with 6GB of GDDR5 memory and 2880 CUDA cores.
%%%%%%%%%%%%%%%%%%%%%%%%%%%%%%%%%%%%%%%%%%%%%%%%%%%%%%%%%%%%%%%%%%%%%%%%%%%%%%%%%%%%%%%
%% CHAPTER
\chapter{Results}
\section{Data}
\section{Error rates per read}
\section{Consensus analysis}

%%%%%%%%%%%%%%%%%%%%%%%%%%%%%%%%%%%%%%%%%%%%%%%%%%%%%%%%%%%%%%%%%%%%%%%%%%%%%%%%%%%%%%%
%% CHAPTER
\chapter{Conclusion}
Conclusion.



%%%%%%%%%%%%%%%%%%%%%%%%%%%%%%%%%%%%%%%%%%%%%%%%%%%%%%%%%%%%%%%%%%%%%%%%%%%%%%%%%%%%%%%
%% DONE
\bibliography{references}
\bibliographystyle{plainnat}

\begin{abstract}
Abstract.

\keywords{Keywords.}
\end{abstract}

% TODO: Navedite naslov na hrvatskom jeziku.
\hrtitle{Model dubokog učenja za određivanje očitanih baza dobivenih uređajem za sekvenciranje MinION}
\begin{sazetak}
Sažetak na hrvatskom jeziku.

\kljucnerijeci{Ključne riječi, odvojene zarezima.}
\end{sazetak}

\end{document}
